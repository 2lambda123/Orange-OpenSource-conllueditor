\documentclass[12pt]{article}

\usepackage{fontspec}
\setmainfont{Times New Roman}

\usepackage{deptree}
\usepackage{xcolor}
\usepackage{float}
\usepackage{metalogo}
\usepackage[colorlinks=true,
            urlcolor=urlcol]{hyperref}
\urlstyle{same}
\definecolor{urlcol}{rgb}{0,0,0.75}

\title{Generating simple Dependency Trees with \XeLaTeX}
\author{Johannes Heinecke\\johannes.heinecke@orange.fr}
\begin{document}

\maketitle


\section{Introduction}
This package generates simple dependency trees (in contrast to
dependency graphs produced by {\tt tikz-dependency package}).
It depends on {\tt tikz} and some of its libaries ({\tt arrows}, {\tt positioning})

The package is loaded by \verb:\usepackage{deptree}:. Do not forget to
load \verb:\usepackage{xcolor}: to be able to redefine colours


Dependency tries are technically speaking a \verb:tikzpicture:, so
all tree commands must be in an \verb:tikzpicture:-environment. The
size of the horizontal and vertical steps can be given by the
\verb:tikz:-options \verb:x=: and \verb:y=::


\begin{verbatim}
   \begin{tikzpicture}[x=20mm,y=20mm]
   \end{tikzpicture}
\end{verbatim}


\section{Declaring dependency relations}
The root is declared by {\tt\char92root\{\emph{word position}\}\{\emph{form}\}\{\emph{POS}\}}. 
The POS field can be split into several lines (e.g.to give UPOS and
XPOS)) by {\verb:\\:}.
Non root word forms are given by 

{\tt\char92dep\{\emph{head pos}\}\{\emph{word pos}\}\{\emph{vertical pos}\}\{\emph{form}\}\{\emph{POS}\}\{\emph{dep rel}\}}.

\begin{itemize}
\item \verb:head pos: is the horizontal position of the head in the sentence
\item \verb:word pos: is the horizontal position of the dependent in the sentence
\item \verb:vertical pos: is the vertical position of the dependent
  (counted from the root)
\end{itemize}

A bottom line which repeats the words of the sentence, can be added by
specifying the vertical position of this line
{\tt\char92setbottom\{\emph{vert pos}\}}. This value will be used for
all following trees. So it either has to be set to an correct value
for each sentence or to 0 to deactivate the bottom line.

For instance, the code in figure \ref{def1} results in figure \ref{ex1}:

\begin{figure}[H]
\begin{verbatim}
  \setbottom{4} % set to 0 to hide bottom line of forms
  \begin{tikzpicture}[x=20mm,y=20mm]
  \root{2}{gefais}{VERB}
  \dep{2}{1}{2}{Mi}{PART}{advmod}
  \dep{2}{3}{2}{i}{PRON}{nsubj}
  \dep{2}{5}{2}{ngheni}{NOUN}{ccomp}
  \dep{2}{7}{2}{Nghaerdydd}{PROPN}{obl}
  \dep{2}{8}{2}{.}{PUNCT}{punct}
  \dep{5}{4}{3}{fy}{PRON}{nmod:obj}
  \dep{7}{6}{3}{yng}{ADP}{case}
  \end{tikzpicture}
\end{verbatim}
\caption{dependency tree definition}\label{def1}
\end{figure}

The CoNLL-U editor
(\url{https://github.com/Orange-OpenSource/conllueditor}) generates
the tree definitions automatically from any CoNLL-U file.





\def\beispiel{
\setbottom{3} % set to 0 to hide bottom line of forms
\begin{tikzpicture}[x=20mm,y=20mm]
\root{2}{gefais}{VERB}
% headpos,xpos,ypos,form,pos,label
\dep{2}{1}{1}{Mi}{PART}{advmod}
\dep{2}{3}{1}{i}{PRON}{nsubj}
\dep{2}{5}{1}{ngheni}{NOUN}{ccomp}
\dep{2}{7}{1}{Nghaerdydd}{PROPN}{obl}
\dep{2}{8}{1}{.}{PUNCT}{punct}
\dep{5}{4}{2}{fy}{PRON}{nmod:obj}
\dep{7}{6}{2}{yng}{ADP}{case}
\end{tikzpicture}
}


\begin{figure}[H]
\beispiel
\caption{Default dependency tree}\label{ex1}
\end{figure}

\section{Customisation}
The package has limited configuration options:
\begin{itemize}
\item \verb:\setdeprelcolor{blue}: color of the lines linking head and
  dependant
\item \verb:\setdeprellabelcolor{red}: colour of the rectangle
  containing the dependency relation name
\item \verb:\setdeprelbgcolor{blue!15}: color of the background of the
  dependency relation name
  
\item \verb:\setdepreltextcolor{violet}:colour of the dependency relation name
\item \verb:\setdepreltextfont{\sf}: font for the dependency relation  name
\item \verb:\setwordfont{\sf}: font for word nodes
\item \verb:\setbottomwordfont{\sf}: font for words on the bottom line
\item \verb:\setmtwfont{\it\scriptsize}: font for multiword token (bottom line)
\item \verb:\setmtwlabelcolor{blue!10}: background colour for multiword token
\end{itemize}

The \verb:set*font: commands accept any font command, including new
fonts declared with \XeLaTeX' \verb:fontspec:.

For instance, adding the the customization shown in figure \ref{def2}
\emph{before} 
the tree definition, results in figure~\ref{ex2}:

\begin{figure}[H]
\begin{verbatim}
\setdeprelcolor{blue}
\setdeprellabelcolor{black}
\setdepreltextcolor{blue!50!black}
\setdeprelbgcolor{blue!15}    
\setdepreltextfont{\it\scriptsize}
\setwordfont{\large\sf}
\setbottomwordfont{\footnotesize\sf}
\setmtwfont{\it\scriptsize}
\setmtwlabelcolor{blue!10}
\end{verbatim}
\caption{customization}\label{def2}
\end{figure}



\setdeprelcolor{blue}
\setdeprellabelcolor{black}
\setdepreltextcolor{blue!50!black}
\setdeprelbgcolor{blue!15}
\setdepreltextfont{\it\scriptsize}
\setwordfont{\large\sf}
\setbottomwordfont{\footnotesize\sf}
\setmtwfont{\it\scriptsize}
\setmtwlabelcolor{blue!10}



\begin{figure}[H]
\beispiel
\caption{Customized dependency tree}\label{ex2}
\end{figure}



The vertical position can be used to modify the layout of the
trees. For instance if we change the line from figure \ref{def1}
to \verb+\dep{5}{4}{2.3}{fy}{PRON}{nmod:obj}+
 will lower the node of
word \emph{fy} slightly (cf. figure \ref{ex3}).



\begin{figure}[H]
\setbottom{3} % set to 0 to hide bottom line of forms
\begin{tikzpicture}[x=20mm,y=20mm]
\root{2}{gefais}{VERB}
% headpos,xpos,ypos,form,pos,label
\dep{2}{1}{1}{Mi}{PART}{advmod}
\dep{2}{3}{1}{i}{PRON}{nsubj}
\dep{2}{5}{1}{ngheni}{NOUN}{ccomp}
\dep{2}{7}{1}{Nghaerdydd}{PROPN}{obl}
\dep{2}{8}{1}{.}{PUNCT}{punct}
\dep{5}{4}{2.3}{fy}{PRON}{nmod:obj}
\dep{7}{6}{2}{yng}{ADP}{case}

\end{tikzpicture}
\caption{Vertical position modified}\label{ex3}
\end{figure}


\section{Multiword tokens}
Multiword tokens (MWT) can be indicated by a \verb:\mwt: command
(e.g. figure \ref{ex4}):

 {\tt\char92mwt\{\emph{first word pos}\}\{\emph{last word
     pos}\}\{\emph{MWT}\}} 



\begin{figure}[H]
\setbottom{3} % set to 0 to hide bottom line of forms
\begin{tikzpicture}[x=20mm,y=20mm]
\root{1}{mus\'ee}{NOUN}
\dep{1}{4}{1}{Louvre}{PROPN}{nmod}
\dep{4}{2}{2}{de}{ADP}{case}
\dep{4}{3}{2}{le}{DET}{det}
\mwt{2}{3}{du}
\end{tikzpicture}
\caption{Multiword tokens}\label{ex4}
\end{figure}



Small trees can be obtained by modifying the \verb:x=:/\verb:y=:
options. Do not chose values to small (as I did here) without lowering
font size accordingly, since the
deprel labels will overlap with nodes.

\begin{figure}[H]
\begin{tikzpicture}[x=12mm,y=15mm]
\root{2}{gefais}{VERB}
% headpos,xpos,ypos,form,pos,label
\dep{2}{1}{1}{Mi}{PART}{advmod}
\dep{2}{3}{1}{i}{PRON}{nsubj}
\dep{2}{5}{1}{ngheni}{NOUN}{ccomp}
\dep{2}{7}{1}{Nghaerdydd}{PROPN}{obl}
\dep{2}{8}{1}{.}{PUNCT}{punct}
\dep{5}{4}{2.3}{fy}{PRON}{nmod:obj}
\dep{7}{6}{2}{yng}{ADP}{case}
\end{tikzpicture}
\caption{Small tree}\label{ex4}
\end{figure}











\iffalse
\setbottom{3}
\begin{tikzpicture}[x=20mm,y=20mm]
%  \node [wort] (root) at(5,0) {Mae\\VERB};
 \root{5}{mae}{VERB}
% \depl{1}{root}{y}{Y\\PART}{advmod}
% \depr{1}{root}{s}{Si\^on\\PROPN}{nsubj}
% \depr{1}{s}{b}{bach\\ADJ}{amod}
% \depr{5}{root}{f}{fach\\ADJ}{amod}
% \depl{1}{f}{y}{yn}{PART}{amod}
 \dep{5}{4}{1}{Y}{PART}{advmod}
 \dep{5}{6}{1}{Si\^on}{PROPN}{nsubj}
 \dep{6}{7}{2}{bach}{ADJ}{amod}
 \dep{5}{9}{1}{fach}{ADJ}{amod}
 \dep{9}{8}{2}{yn}{PART}{amod}

 \root{10}{gwelais}{VERB}
 \dep{10}{11}{1}{i}{PRON}{nsubj}

%  \node [wort, below left=1 and 0 of root] (mw)  {B\\ART}; 
%  \path [line] (root) -> node [sloped,midway,above] {ss} (mw);
%  \node [wort, below left=1 and 1 of root] (mw2)  {C\\DEF}; 
%  \path [line] (root) -> node [sloped,midway,above] {mw2} (mw2);


%  \node [wort] (mw2)  at (3,-1) {C\\DEF}; 
%  \path [line] (root) -> node [sloped,midway,above] {eeee} (mw2);
%  \node [wort] (mw3)  at (3,-2) {Q}; 
%  \node [wort] (mw4)  at (2,-1) {DDDD}; 
\end{tikzpicture}


\setbottom{6}
\begin{tikzpicture}[x=15mm,y=20mm]
\root{5}{falch}{ADJ}
% headpos,xpos,ypos,form,pos,label
\dep{5}{1}{2}{Yr}{PART}{qadvmod}
\dep{5}{2}{2}{ydyn}{AUX}{cop}
\dep{5}{3}{2}{ni}{PRON}{nsubj}
\dep{5}{4}{2}{'n}{AUX}{aux}
\dep{5}{7}{2}{bod}{NOUN}{ccomp}
\dep{7}{6}{3}{ein}{PRON}{nmod:nsubj}
\dep{7}{9}{3}{bod}{NOUN}{xcomp}
\dep{9}{8}{4}{wedi}{AUX}{aux}
\dep{9}{11}{4}{trio}{NOUN}{xcomp}
\dep{11}{10}{5}{yn}{AUX}{aux}
\end{tikzpicture}






\begin{tikzpicture}[x=15mm,y=20mm]
\root{5}{falch}{ADJ}
\dep{5}{1}{1}{yr}{PART}{advmod}
\dep{5}{2}{1}{ydyn}{AUX}{cop}
\dep{5}{3}{1}{ni}{PRON}{nsubj}
\dep{5}{4}{1}{'n}{PART}{c:pred}
%\dep{5}{4}{1}{. . .}{}{. . .}
\dep{5}{7}{1}{bod}{NOUN}{ccomp}
\dep{7}{6}{2.3}{ein}{PRON}{n:nsubj}
\dep{7}{9}{2}{bod}{NOUN}{xcomp}
\dep{9}{8}{3.3}{wedi}{AUX}{aux}
\dep{9}{11}{3}{trio}{NOUN}{xcomp}
\dep{11}{10}{4.1}{yn}{AUX}{aux}
\end{tikzpicture}
\fi

\end{document}
